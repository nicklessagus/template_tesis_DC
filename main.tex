\documentclass[11pt,a4paper,twoside]{tesis}
% SI NO PENSAS IMPRIMIRLO EN FORMATO LIBRO PODES USAR
%\documentclass[11pt,a4paper]{tesis}

%para forzar algunas imágenes que vayan donde yo quiero con el [H]
\usepackage{float}

%Para rellenar con lorem ipsum, obvio que se puede sacar
\usepackage{lipsum}  

%para decirle a graphicx donde ir a buscar las imágenes y safar de poner la ruta
\usepackage{graphicx}
\graphicspath{{img/}{../img/}}

\usepackage[utf8]{inputenc} %por si se usa pdflatex
\usepackage[spanish,es-nodecimaldot]{babel} %importante el tema del punto decimal
\usepackage[left=3cm,right=3cm,bottom=3.5cm,top=3.5cm]{geometry}

%para poner texto en un math con \text{}
\usepackage{amsmath}

%links coloridos en el PDFs
%queda horrible el rojo, pero para que se note
\usepackage{hyperref}
\hypersetup{
	colorlinks=true,
	linkcolor=red,
	urlcolor=blue,
	citecolor=red
}

\usepackage{amsmath}
\usepackage{algorithm}
\usepackage[noend]{algpseudocode}

%%%%%%%%%%%%%%%%%%%%%%%%%%%%%%%
\usepackage{todonotes}
\setlength{\marginparwidth}{25mm}

%% El paquete TODO es para poner notas o reservar espacio para imágenes. 
%% Ejemplos: 
%% \todo{Make a cake}
%% \missingfigure
%% \missingfigure{poner dibujo de X}

%%%%%%%%%%%%%%%%%%%%%%%%%%%%%%%
% Para poner el watermark de DRAFT. 
% las macros son para poder poner y sacar en cada capítulo a medida que se vayan terminando
\usepackage{draftwatermark}
\SetWatermarkText{DRAFT}
\SetWatermarkScale{2}

\makeatletter
\let\originalsc@watermark\sc@watermark
\newcommand{\draftwatermarkon}{%
  \let\sc@watermark\originalsc@watermark
}
\newcommand{\draftwatermarkoff}{%
  \let\sc@watermark\@empty
}
%%%%%%%%%%%%%%%%%%%%%%%%%%%%%%%
\usepackage{listings}
\usepackage{xcolor}

%%%%%%%%%%%%%%%%%%%%%%%%%%%%%%%
%%% Para colorear código
\definecolor{codegreen}{rgb}{0,0.6,0}
\definecolor{codegray}{rgb}{0.5,0.5,0.5}
\definecolor{codepurple}{rgb}{0.58,0,0.82}
\definecolor{backcolour}{rgb}{0.95,0.95,0.92}

\lstdefinestyle{mystyle}{
	backgroundcolor=\color{backcolour},   
	commentstyle=\color{codegreen},
	keywordstyle=\color{magenta},
	numberstyle=\tiny\color{codegray},
	stringstyle=\color{codepurple},
	basicstyle=\ttfamily\footnotesize,
	breakatwhitespace=false,         
	breaklines=true,                 
	captionpos=b,                    
	keepspaces=true,                 
	numbers=left,                    
	numbersep=5pt,                  
	showspaces=false,                
	showstringspaces=false,
	showtabs=false,                  
	tabsize=2
}

\lstset{style=mystyle}
\linespread{1.3}	

%%%%%%%%%%%%%%%%%%%%%%%%%%%%%%%

% Para poder trabajar cada capitulo/archivo por separado
% si compilo el main incluye los .tex de los capitulos como es habitual 
% pero tambien puedo compilar los capitulos por separado para poder
% concentrarse en cada uno sin tener que trabajar con todo el documento

% IMPORTANTE:
% hay que hacer link simbolicos al main, tesis y bibliography en el directorio chapters
%
% ln -s ../bibliography.bib bibliography.bib
% ln -s ../tesis.cls tesis.cls
% ln -s ../main.tex main.tex

% Macro para que corra comandos solo si es compilado el subfile standalone
\usepackage{subfiles}
\newcommand{\onlyinsubfile}[1]{#1}


\begin{document}
\renewcommand{\onlyinsubfile}[1]{}

%%%%%%%%%%%%%%%%%%%%%%5
%%% Si se quiere poner o sacar el draft de la portada y las primeras secciones
\draftwatermarkoff



%%%% CARATULA
\def\autor{Cosme Fulanito}
%Título de la carátula
\def\tituloTesis{Buenas noches señores, ¿molesto con una copilla, por favor?}
%Títulos del abstract en español/inglés
\def\runtitulo{Buenas noches señores, \\¿molesto con una copilla, por favor?}
\def\runtitle{Greetings, good man. \\ Might I trouble you for a drink?}
\def\director{Dr. Nick Riviera}
\def\codirector{Dr. Hibbert}
\def\lugar{Springfield, 2020}
\input{caratula}

%%%% ABSTRACTS, AGRADECIMIENTOS Y DEDICATORIA
\frontmatter
\pagestyle{empty}
%\begin{center}
%\large \bf \runtitulo
%\end{center}
%\vspace{1cm}
\chapter*{\runtitulo}

\noindent
Eran los mejores tiempos, eran los peores tiempos, era el siglo de la locura, era el siglo de la razón, era la edad de la fe, era la edad de la incredulidad, era la época de la luz, era la época de las tinieblas, era la primavera de la esperanza, era el invierno de la desesperación, lo teníamos todo, no teníamos nada, íbamos directos al Cielo, íbamos de cabeza al Infierno; era, en una palabra, un siglo tan diferente del nuestro que, en opinión de autoridades muy respetables, solo se puede hablar de él en superlativo, tanto para bien como para mal.
\bigskip

\noindent\textbf{Palabras claves:} Por, Fin, Me, Recibo.


\cleardoublepage
\input{abs_en.tex} % OPCIONAL: comentar si no se quiere 

\cleardoublepage
\chapter*{Agradecimientos}

\noindent A todos los que me conocen.

\vspace{0.5cm}

\noindent A los que no, también.
 % OPCIONAL: comentar si no se quiere

\cleardoublepage
\begin{flushright}
	\begin{footnotesize}
		
		\textit{No sabes como es eso Marge, Yo soy el que va ahí a romperse el alma. Y no estoy fuera de lugar, Tu estas fuera de lugar, Todo el maldito sistema esta fuera de lugar, Quieres la verdad ?? Tu no puedes manejar la verdad Porque cuando se levanta la mano para tocar la cara de lo que fue tu mejor amigo y es un montón de basura, Uno no sabe que hacer... OLVÍDALO MARGE, ESTO ES EL BARRIO CHINO!!}\\        
		-- Homero J. Simpson, S05E22
		
	\end{footnotesize}
\vspace{1cm}
A encias sangrantes Murphy.
\end{flushright}

  % OPCIONAL: comentar si no se quiere

\cleardoublepage
\tableofcontents

\mainmatter
\pagestyle{headings}

%Si se quiere la lista de todos los TODO del documento
%\listoftodos[Notes]

%%%% ACA VA EL CONTENIDO DE LA TESIS
\subfile{chapters/chapter-1}
\newpage
\subfile{chapters/chapter-2}
\newpage
\subfile{chapters/chapter-3}
\newpage
\subfile{chapters/chapter-4}
\newpage

%%% BIBLIOGRAFIA
%% Primero va la bibliografia, después el apéndice
%\backmatter %esto está en el template del DC pero ya no va

%Se puede cambiar el estilo de la bibliografia, este es medio críptico :P
\bibliographystyle{alpha}
\bibliography{bibliography}

\appendix
\subfile{chapters/chapter-5}

\end{document}
